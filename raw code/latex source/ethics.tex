\chapter{Ethics}
\label{chap:intro}

The first part of this analysis was getting approval for the data.  Working with the Access to Justice Lab in Harvard Law School, we thought this part would be straight forward, but it turned out that we needed to get approval from an Institutional Review Board (IRB) committee.  Because the subjects of the research are humans, it was deemed necessary to do a review.  All human subjects research requires an IRB ethics review to safeguard against advancing research at a cost of abuse or harm to research subjects.

\par We applied for an exemption based on the grounds that the information from the courts was publicly accessible, the information from the census demographics was publicly available, and the surname data from the census was also publicly available.  Nothing that we were using was data that was otherwise hidden or considered sensitive.  As well, the methodology we wanted to use, the Bayesian improved surname geocoding (BISG), had been used before and the technique itself was publicly published.  The difference between the previous work and the present work was that we would be using the BISG algorithm on housing records and not on medical or automotive data.  That said, research has and can go afoul, our exemption request was initially declined, and we were required to apply for approval from the IRB committee through Harvard Law school. \par

Applying for approval first consisted of completing the IRB training.  I was the only person who would gain access to a pre-downloaded archive of court data who didn't have prior IRB training and approval, so that was the first immediate hurdle, IRB training through Citi Program~\citep{WEBSITE:9}.  Training consisted of reading and understanding various modules on ethics as they relate to the study at hand.  The first module, required for anyone seeking approval for working with human subjects, gives examples of famous violations of ethics in the past, and uses these to motivate the three principles we come back to repeatedly: Respect for persons, Beneficence, and Justice.  In understanding each of these, we are given the definitions.\\
Respect for Persons:
\begin{quote}
“Respect for persons incorporates at least two ethical convictions: first, that individuals should be treated as autonomous agents, and second, that persons with diminished autonomy are entitled to protection. The principle of respect for persons thus divides into two separate moral requirements: the requirement to acknowledge the autonomy and the requirement to protect those with diminished autonomy." ~\citep{WEBSITE:10}
\end{quote}
Elaborating on autonomy:
\begin{quote}
“Autonomy means that people must be empowered to make decisions concerning their own actions and well-being. According to the principle of respect for persons, researchers must acknowledge the "considered opinions and choices" of research subjects. In other words, individuals must be given the choice whether to participate in research, and they must be provided sufficient information and possess the mental competence to make that choice.” ~\citep{WEBSITE:10}
\end{quote}

Certainly it is the case that this research would be studying human subjects, and we would not be sending informed consent forms to everyone who had been evicted in the state of Massachusetts.  Nor would we be taking special consideration for those who had diminished autonomy, children or adults of limited autonomy.  Considering this, we push on to the other two components that had to be considered in human subject research, beneficence and justice.
Elaborating on Beneficence:
\begin{quote}
“Persons are treated in an ethical manner not only by respecting their decisions and protecting them from harm, but also by making efforts to secure their well-being. Such treatment falls under the principle of beneficence. The term 'beneficence' is often understood to cover acts of kindness or charity that go beyond strict obligation. In this document, beneficence is understood in a stronger sense, as an obligation. Two general rules have been formulated as complementary expression of beneficent actions in this sense: (1) do not harm and (2) maximize possible benefits and minimize possible harms.” ~\citep{WEBSITE:10}
\end{quote}
\par
Here, we seemed to be on good footing.  The potential use case of this research is to restrict access to eviction records if they are harming a protected class of citizens.  The supposition going into this research is that landlords are using housing eviction records to not rent to minorities.  Because it is hard to prove this and can not really be halted, the goal is to assess the harm being done preceding this claim with the initial eviction.  If it can be shown that the original evictions are done in a manner which is biased against minority populations, legal action regarding disparate impact could allow for the sealing of eviction record (so they would no longer be publicly accessible, akin to divorce records which are deemed to not be of public interest, or juvenile records which are sealed to safeguard a protected class of citizens).  Point being, the goal was to show harm already exists, and remove the harm from happening, both directly and indirectly by measuring potential bias, and barring future difficulty in finding housing.  If the research reaches no conclusive results, no harm is done.  If it does reach conclusive results, either nothing or beneficence.
The last major component in the IRB training, Justice:
\begin{quote}
“Who ought to receive the benefits of research and bear its burdens? This is a question of justice, in the sense of "fairness in distribution" or "what is deserved." An injustice occurs when some benefit to which a person is entitled is denied without good reason or when some burden is imposed unduly.” ~\citep{WEBSITE:10}
\end{quote}
\par
Justice in this context is seen as equitable distribution of both benefit and burden.  This is explained in the training through the context of medical trials and the like, where people may be sick and might be in the control or test populations of new treatments.  If one group is untreated, or another group gets a poisonous medicine, there is a clear burden to be carried by a subset of the group.  As might happen with medical research, if the test populations are not constructed judiciously, the benefits may only apply to Caucasian males, or children, or some other selective group, therefore inequitably spreading benefit. \par

All the above considerations apply, but most germane to the study in question is the amount of research time consumed.  The needed materials are publicly available through census data, or Massachusetts housing court records.   Acquiring the latter takes more time, but it is still free.  One advisor raised the question of harm that could be done to eviction subjects.  Would it be possible that this technique for unmasking ethnicity be made public and amount to a loss of privacy for the “test subjects”.  While there is this point of potential loss of privacy for these subjects, there is nothing to stop a motivated individual from using the techniques to do this already.  Zip code level demographics are already trivial to acquire.  Drilling down to neighborhoods with census tract data can be done on census.gov.  Finding surname ethnic breakdowns is a bit more tedious than neighborhood level data, but still straightforward.  Given that the tools are readily available, there is no real loss of previously held privacy here.  A curious note is that the the MA court records are not generally searchable in bulk.  The court system is leery of research being done on incomplete or erroneous data.  This does present an issue, but it only presents an issue of harm if the research results in eviction records being obscured from those who would use it.  Arguing that this is a comparatively small population, and a small burden at that, particularly when credit reports could achieve similar assessments on small scale, the incurred burden appears to be minimal. \par
Further breakdown of the ethics review dials in to two points.  The above points are all important, and presented to anyone who might be doing research on human subjects. What then exactly is the definition of “research” and “human subjects” as it applies in this academic setting?
Research, as defined by the common law relevant to the IRB:
\begin{quote}
“Systematic investigation, including research development, testing, and evaluation, designed to develop or contribute to generalizable knowledge.” ~\citep{WEBSITE:11}
\end{quote}
Contrasting with a description of things which are not research:
\begin{quote}
“Scholarly and journalistic activities (for example, oral history, journalism, biography, literary criticism, legal research, and historical scholarship).  Public health surveillance activities.  Collection and analysis of information, biospecimens, or records by or for a criminal justice agency for activities authorized by law or court order or criminal investigative purposes.  Authorized intelligence, homeland security, defense, or national security mission operational activities” ~\citep{WEBSITE:11}
\end{quote}
It appears that this project squarely fits under the heading of research.  While there are legal records analyzed, plausibly for the sake of criminal justice, at the core, the goal is to understand the workings of the system through the use of statistical techniques.  It is a systematic investigation of the Massachusetts housing eviction records, and it is intended to expand the understanding of how the housing system works relative to the underlying population. \par
Having qualified as research, does this research qualify as being on human subjects, as certainly the records are about things which have happened to humans:
\begin{quote}
"Human subject" means a living individual about whom a researcher (whether professional or student) conducting research: Obtains information or biospecimens through intervention or interaction with the individual, and uses, studies, or analyzes the information or biospecimens; or obtains, uses, studies, analyzes, or generates identifiable private information or identifiable biospecimens. ~\citep{WEBSITE:9}
\end{quote}
\par
This is essentially where the need for the IRB becomes moot.  The research being done here isn’t about human subjects.  On the first point, there are no biospecimens.  While there is information, it is not gained through interaction or intervention with the relevant individuals.  On the second point, none of the information which is used is private information.  It could be argued that we are generating or identifying private information (not biospecimens), but to that point, we only create probabilistic models of what a person’s ethnicity is likely to be.  There is no statement of what a person’s ethnicity is, and through repeated sampling, any individual's ethnicity is likely to be estimated to be in multiple different categories, if not eventually every different possible category. \par

Ultimately, this position is one that was agreed upon by the Institutional Review Board committee.  Upon the initial request with the IRB, they asserted that this might be human subjects work and required everyone exposed to the data to have done an IRB training.  But, after having reviewed the proposal and seeing that everyone had done the IRB training, the eventual ruling was that the work was not “human subjects” work, and did not actually need IRB approval (IRB17-1931).  The research was free to progress.
