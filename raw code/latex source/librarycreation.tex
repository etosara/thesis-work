
\chapter*{Library creation and code documentation}

\par In cleaning and processing this data, I’ve written numerous lines of code and variants of functions, especially in the data analysis and plotting.  While first drafts of code were written in the main environment and copied and pasted where needed, this eventually became overwhelming.  When starting a new a analysis, it was untenable to scroll through previous code and find what was needed while also keeping track of what dependencies previous code possessed.  Particularly with different variations of the same function, some merely running faster than others because of a tweak to the middle of the body of the function, it became imperative to create a library of functions.  Once the library was created, it was straightforward to scroll through a few lines of code, instead of blocks of code, and reason about what the results were.  It also facilitated quick plotting and documenting the plots for incorporation within the write-up.

\par Function creation handles the vast majority of code simplification, but an extra advantage of packaging everything into a library is that exports of the for distributing computation then becomes simplified.
